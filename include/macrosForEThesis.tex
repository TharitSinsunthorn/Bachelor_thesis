%%%%%%%%%%%%%%%%%%%%%%%%%%%%%%%%%%%%%%%%%%%%%%%%%%%%%%%%%%%%%%%%%%%%%%
%%
%%   macrosForEThesis.tex
%%   -------------------
%%
%%   Title: LaTeX macro file for the English Thesis
%%   
%%   
%%   Belief:
%%   1."Fig. 1: ~~~", "Table 1: ~~~" are for the figure and table citation.
%%   2. add the config to read the source code file (hello.c) and visualize it. 
%%
%%   Change Log:
%%   2020.11.26 Initial creation by Kentaro UNO
%%
%%   Contact:
%%   Please contact the author Kentaro UNO (unoken.astro@gmail.com) if
%%   you have a problem
%%
%%%%%%%%%%%%%%%%%%%%%%%%%%%%%%%%%%%%%%%%%%%%%%%%%%%%%%%%%%%%%%%%%%%%%%

%%% useful command
\newcommand{\n}{\nonumber \\}
\newcommand{\p}{\partial}
\newcommand{\bs}[1]{\boldsymbol{#1}}
\newcommand{\II}{I\negthinspace I}
\newcommand{\III}{I\negthinspace I\negthinspace I}
\newcommand{\mbm}[1]{\mbox{\protect \boldmath $#1$}}

%%%%%%%%%%%%%%%%%%%%%%%%%%%%%%%%%%%%%%%%
%%%  config for citation of figures, tables, and equations
%%%  to use English for all. 
%%%  2020.01.29 modified by Kentaro UNO
%%%%%%%%%%%%%%%%%%%%%%%%%%%%%%%%%%%%%%%%

%%% figure and table
\captionsetup{compatibility=false}
%
\renewcommand{\figurename}{Fig.~\hspace{-.2em}}   
\renewcommand{\tablename}{Table~\hspace{-.2em}}

\captionsetup[figure]{format=plain,labelformat=simple,labelsep=colon,font=small}
\captionsetup[table]{format=plain,labelformat=simple,labelsep=colon,font=small}

\newcommand{\bhline}[1]{\noalign{\hrule height #1}} % bhlineコマンドの設定

%%% citation command for figures, tables, and equations
\newcommand{\fig}[1]{Fig.~\ref{#1}}
\newcommand{\subfig}[2]{Fig.~\ref{#1}\subref{#2}}
\newcommand{\fign}[1]{\ref{#1}}
\newcommand{\tb}[1]{Table~\ref{#1}}
\newcommand{\tabn}[1]{\ref{#1}}
\newcommand{\eq}[1]{Eq.~(\ref{#1})}
\newcommand{\eqn}[1]{(\ref{#1})}

%%% citation command for chapters, sections, and appendixes
\newcommand{\chap}[1]{Chap.~\ref{#1}}
\newcommand{\sect}[1]{Sect.~\ref{#1}}
\newcommand{\subsect}[1]{Sect.~\ref{#1}}
\newcommand{\app}[1]{Appnd. ~\ref{#1}}

%%% visualize a figure and table in a row
\makeatletter
\newcommand{\figcaption}[1]{\def\@captype{figure}\caption{#1}}
\newcommand{\tblcaption}[1]{\def\@captype{table}\caption{#1}}
\makeatother

%%%%%%%%%%%%%%%%%%%%%%%%%%%%%%%%%%%%%%%%
%%%  ソースコードの引用スタイルの設定
%%%  2020.01.24 added by Kentaro UNO
%%%%%%%%%%%%%%%%%%%%%%%%%%%%%%%%%%%%%%%%
\usepackage{listings} %日本語のコメントアウトをする場合はjlistingが必要
%ここからソースコードの表示に関する設定
\definecolor{gray}{rgb}{0.9,0.9,0.9}
\lstset{
  backgroundcolor=\color{gray}, % choose the background color; you must add \usepackage{color} or \usepackage{xcolor}; should come as last argument
  basicstyle={\small\ttfamily}, % the style of the fonts that are used for the code
  identifierstyle={\small}, % main()の''main''等のスタイル
  commentstyle={\small\itshape}, %//comment 等のスタイル
  keywordstyle={\small}, % int mainの''int''等のスタイル
  ndkeywordstyle={\small},
  stringstyle={\small\ttfamily},  % string literal style -> % printf(''Hello'')の''''で囲まれた部分のスタイル
  %frame={tb},
  breaklines=true, % sets automatic line breaking
  columns=[l]{fullflexible},
  numbers=left, % where to put the line-numbers; possible values are (none, left, right)
  numberstyle={\scriptsize}, % the style that is used for the line-numbers
  stepnumber=1, % the step between two line-numbers. If it's 1, each line will be numbered
  numbersep=1zw, % 行数字とコードの間のマージン
  xrightmargin=0zw, % 表示個所の右側マージン
  xleftmargin=2zw,  % 表示個所の左側マージン
  rulecolor=\color{black},         % if not set, the frame-color may be changed on line-breaks within not-black text (e.g. comments (green here))
  showspaces=false,                % show spaces everywhere adding particular underscores; it overrides 'showstringspaces'
  showstringspaces=false,          % underline spaces within strings only
  showtabs=false,                  % show tabs within strings adding particular underscores
  lineskip=-0.5ex % codeの行間の設定
}

\def\epsgaiji#1{\leavevmode\kern-0.025zw\raise-.37zh\hbox{%
  \epsfile{file=#1,width=1.05zw}}\kern-0.025zw}
%\newcommand{\dfrac}[2]{\frac{\displaystyle{#1}}{\displaystyle{#2}}}
\newcommand{\MARU}[1]{{\ooalign{\hfil#1\/\hfil\crcr\raise.167ex\hbox{\mathhexbox20D}}}}

%%%